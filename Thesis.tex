% Options for packages loaded elsewhere
\PassOptionsToPackage{unicode}{hyperref}
\PassOptionsToPackage{hyphens}{url}
%
\documentclass[
]{article}
\usepackage{lmodern}
\usepackage{amssymb,amsmath}
\usepackage{ifxetex,ifluatex}
\ifnum 0\ifxetex 1\fi\ifluatex 1\fi=0 % if pdftex
  \usepackage[T1]{fontenc}
  \usepackage[utf8]{inputenc}
  \usepackage{textcomp} % provide euro and other symbols
\else % if luatex or xetex
  \usepackage{unicode-math}
  \defaultfontfeatures{Scale=MatchLowercase}
  \defaultfontfeatures[\rmfamily]{Ligatures=TeX,Scale=1}
\fi
% Use upquote if available, for straight quotes in verbatim environments
\IfFileExists{upquote.sty}{\usepackage{upquote}}{}
\IfFileExists{microtype.sty}{% use microtype if available
  \usepackage[]{microtype}
  \UseMicrotypeSet[protrusion]{basicmath} % disable protrusion for tt fonts
}{}
\makeatletter
\@ifundefined{KOMAClassName}{% if non-KOMA class
  \IfFileExists{parskip.sty}{%
    \usepackage{parskip}
  }{% else
    \setlength{\parindent}{0pt}
    \setlength{\parskip}{6pt plus 2pt minus 1pt}}
}{% if KOMA class
  \KOMAoptions{parskip=half}}
\makeatother
\usepackage{xcolor}
\IfFileExists{xurl.sty}{\usepackage{xurl}}{} % add URL line breaks if available
\IfFileExists{bookmark.sty}{\usepackage{bookmark}}{\usepackage{hyperref}}
\hypersetup{
  pdftitle={What are the effective features of consultation? A mixed methods approach},
  hidelinks,
  pdfcreator={LaTeX via pandoc}}
\urlstyle{same} % disable monospaced font for URLs
\usepackage[margin=1in]{geometry}
\usepackage{longtable,booktabs}
% Correct order of tables after \paragraph or \subparagraph
\usepackage{etoolbox}
\makeatletter
\patchcmd\longtable{\par}{\if@noskipsec\mbox{}\fi\par}{}{}
\makeatother
% Allow footnotes in longtable head/foot
\IfFileExists{footnotehyper.sty}{\usepackage{footnotehyper}}{\usepackage{footnote}}
\makesavenoteenv{longtable}
\usepackage{graphicx,grffile}
\makeatletter
\def\maxwidth{\ifdim\Gin@nat@width>\linewidth\linewidth\else\Gin@nat@width\fi}
\def\maxheight{\ifdim\Gin@nat@height>\textheight\textheight\else\Gin@nat@height\fi}
\makeatother
% Scale images if necessary, so that they will not overflow the page
% margins by default, and it is still possible to overwrite the defaults
% using explicit options in \includegraphics[width, height, ...]{}
\setkeys{Gin}{width=\maxwidth,height=\maxheight,keepaspectratio}
% Set default figure placement to htbp
\makeatletter
\def\fps@figure{htbp}
\makeatother
\setlength{\emergencystretch}{3em} % prevent overfull lines
\providecommand{\tightlist}{%
  \setlength{\itemsep}{0pt}\setlength{\parskip}{0pt}}
\setcounter{secnumdepth}{-\maxdimen} % remove section numbering

\title{What are the effective features of consultation? A mixed methods
approach}
\author{}
\date{\vspace{-2.5em}28/05/2021}

\begin{document}
\maketitle

DO I NEED TO DEFINE THE WORD `FEATURE'?

\hypertarget{introduction}{%
\section{1 Introduction}\label{introduction}}

This research consisted of interviews with Educational Psychologists
(EPs), the development of a novel questionnaire, and observations of
joint school-parent consultations with long-term follow-up. This was in
service of exploring what the core features of consultation are,
according to EP self-report and observation of real-world consultations,
and how EPs altered their practice to adapt to the COVID-19 global
pandemic. The interviews explored EP's definition of consultation, their
views on what the key features of effective consultations are, what some
of the barriers are, how they have changed their consultation practice
as a result of the pandemic and the advantages and disadvantages of
this. This was supplemented by a questionnaire which asked similar
questions, as well as asking participants to identify the different
kinds of work they were engaging in during the pandemic. The observation
schedule was informed by the relevant literature and was used to see how
often different features of consultation were observed during a joint
school-parent consultation and in what order. This was then to be
cross-referenced with reported progress towards jointly agreed goals for
the child and young people (CYP) to see which features correlate with
improved outcomes. This work built on a previous piece of research
exploring what EPs believed were the most important features of
consultation and a thematic analysis of recorded initial consultations
to identify the main features in a live consultation.

\hypertarget{literature-review}{%
\subsection{1.1 Literature Review}\label{literature-review}}

A literature review was conducted to see what previous research had
found to be the main features of consultation and what the main tools of
analysing the efficacy of consultation were. Various databases,
including Web of Science and Scopus, were searched using the key words
``educational psychology'' and ``consultation''. Key references, such as
@kennedy\_educational\_2008, were given to the researcher by their
supervisor to set a baseline for the literature review. Considering all
the relevant literature, there is some consistency around EPs views
regarding consultation. However, there is a heterogeneity of
understanding from other stakeholders as to what consultation actually
means. Crucially, there is a relatively small amount of research
exploring what happens during a consultation
{[}@kennedy\_educational\_2008{]}, as well as few studies evaluating the
efficacy of consultation. There are also few studies which attempt to
analyse what makes consultation effective or what the effective features
of consultation are. This leaves EPs and associated stakeholders with a
widely used but poorly understood and validated framework.

\hypertarget{what-is-consultation}{%
\subsubsection{1.1.1 What is consultation?}\label{what-is-consultation}}

Consultation takes many different forms across contexts and countries.
Consequently, there is not a universal definition of consultation as
conducted by EPs. This raises an important problem for any EP who wishes
to engage in consultation or analyse its efficacy. Within a western
context, it fundamentally involves problem solving between consultants
(EPs) and consultees. The consultee is most often a teacher who knows
the CYP well, but it can also be parents and/or Special Educational
Needs Coordinator (SENCOs). In joint school-family consultations, it is
generally agreed prior to the consultation that at least one member from
the child's family unit and the school will attend. These individuals
collaborate to devise and establish interventions to help support and
find solutions for the client, the CYP {[}@ofarrell\_research\_2018{]}.
Consultation is considered a form of `indirect' work as the theory is
that the EP can enact the most change for the CYP by meeting and working
with those around the CYP {[}@gutkin\_reconceptualizing\_1990{]}. They
may or may not engage one to one with the CYP but it is not mandated by
this approach.

Consultation has become the model of service delivery for many
Educational Psychology Services {[}@sheridan\_randomized\_2017{]}. Most
Educational Psychology Services (EPS) in the U.K. have moved towards a
predominantly consultation-based service
{[}@ofarrell\_research\_2018{]}. This is in contrast with what is viewed
as a more traditional model which predominantly involves individual
casework, typically including the administration of a cognitive
assessment {[}@kratochwill\_evidence\_based\_2002;
@larney\_school\_based\_2003{]}. The most commonly employed consultation
framework in the U.K. is the Wagner model
{[}@wagner\_consultation\_1995; @wagner\_school\_1995;
@wagner\_consultation\_2000{]}. It is defined as ``a voluntary,
collaborative, non-supervisory approach, established to aid the
functioning of a system and as inter-related systems''
{[}@wagner\_consultation\_2000{]} through ``purposeful
{[}conversations{]} which {[}use{]} techniques of listening, clarifying,
problem-solving, challenging, questioning and reflecting''
{[}@munro\_angles\_2000{]}. As a result, EPs work with those closest to
the CYP, but not as experts telling those directly involved with the CYP
how to help them. Their role is to help empower the consultees to solve
their own problems in school. The focus is not only on the CYP but their
relations with others and the many different environments they are in,
such as home, school, and their wider community
{[}@bronfenbrenner\_ecology\_1981{]}. There is an understanding of the
interactions between these layers and the need to consider a child
holistically. This support is provided by asking questions, analysing
presenting problems and helping others think differently, agreeing on
potential interventions, and then reflecting on the whole process so
progress can be made.

\hypertarget{how-prevalent-is-consultation-in-the-u.k.}{%
\subsubsection{1.1.2 How prevalent is consultation in the
U.K.?}\label{how-prevalent-is-consultation-in-the-u.k.}}

The move towards a consultation-based model of service is reflected in
government legislation. The Special Education Needs: Code of Practice
characterises consultation as one of the main services of EPs
{[}@department\_for\_education\_send\_2015{]}. Several studies have also
found it makes up a large percentage of their time working with schools.
@shannon\_educational\_2007 delivered questionnaires to 44 EPs, asking
for the EPs to self-report how often they undertook different types of
work, including consultation or case work. 32 responded, with most
reporting they spent a majority of their time engaging in individual
level work. 91\% of the EPs who were doing individual level work stated
consultation was the main activity performed. However, the authors do
not provide a definition of consultation nor ask the EPs to provide a
definition of consultation. Given that consultation takes many different
forms and there are a wide range of views on consultation between EPs
and other stakeholders, ensuring everyone has the same definition of the
process is crucial. Without it, one cannot be sure different EPs are
engaging in consultation in a similar way and that the schools
understand what they are doing. Participants may have reported they used
consultation, but in practice their methods may be very different.
Because of the limits of self report, we do not know if such a disparity
exists in this study. On the other hand, the EPs who responded were from
a large range of locations across the U.K., thus increasing the
representativeness of the data.

Another study exploring the prevalence of consultation in the UK comes
from @leadbetter\_patterns\_2000. The authors sent questionnaires to all
Principal Educational Psychologists (PEPs) and asked about their models
of service delivery. Consultation was reported as one of the most
frequently used models. However, there was only a return rate of 58\%,
with those not returning almost certainly not randomly distributed.
There is therefore uncertainty around the amount of bias in the results.
If the non-returns were randomly distributed on key variables, such as
whether the EPs has recently experienced a number of organisational
changes, then a low return rate would not introduce bias into the
results. But this is unlikely. As such, the results of certain PEPs who
may have different results from the norm are missing.

Although consultation forms the bedrock of many EPs work and the model
of service delivery for EPS, EPs often struggle to articulate what a
consultation model of service entails. @leadbetter\_role\_2004 argues
there is little research which explicates the structure and process of
consultation. This situation has not improved over the following years,
with the few studies examining this typically only focusing on one EPS
or a small number of EPs {[}@cording\_study\_2011;
@leadbetter\_investigating\_2006; @pipher\_consultation\_2013{]}. These
limitations prohibit one from developing a broad picture of how
consultation is performed in the U.K.

\hypertarget{what-are-consultees-views-on-consultation}{%
\subsubsection{1.1.3 What are consultees views on
consultation?}\label{what-are-consultees-views-on-consultation}}

Before exploring what occurs in a consultation, understanding what those
involved (EPs, teachers) believe it to be is important. This is because
if consultees are to play an active role in consultations (as all models
of consultation state they should), their views of consultation need to
be understood. That way, any misunderstandings can be cleared up and
consultation can be effective. To serve this end, the following section
explore stakeholders' views of consultation as detailed in the relevant
scientific literature. @ofarrell\_research\_2018 interviewed three
groups containing a teacher, an EP, and the parents of the child the
consultation was about. The teachers and parents reported that wile they
appreciated and saw the value of consultation, there was a lack of
understanding regarding its process and nature. All three teachers
implied they viewed the EP as the external expert, who had specialist
knowledge and access to resources which they wanted. This is in direct
contrast to the sentiments expressed by the EPs in this study. They
explicitly stated they were not experts and tried to distance themselves
from that sentiment. This concept is fundamental to many models of
consultation, including Wagner's. This research was conducted in the
Republic of Ireland. Here, consultation has only become the dominant
model of service delivery in recent years. Thus, U.K. based teachers and
SENCOs may have a better understanding. However, few pieces of research
have been conducted to explore understanding of this important strand of
EP work.

@dennis\_so\_2004 interviewed SENCOs at twelve schools to better
understand their views on EP work. One of the core themes raised by the
participants was a wish for EPS's to ``publicise more the range of
things it has to offer and good/innovative practice''. The exact number
of schools which held this belief is unknown as that information was not
reported in the paper. Regarding consultation, there was a large
heterogeneity in the the school's understanding of it. Some schools were
fully aware of the consultation model of service delivery and preferred
it to previous models. Such models focused on direct work, including
assessment of CYP using cognitive assessments. However, other schools,
either did not understand the consultation model or were only partially
aware of it but preferred other models which placed a primacy on
child-focused individual work. They reported they believed the
consultation model involved ``too much talk, not enough action''
{[}@dennis\_so\_2004, 22{]} and thus did not value it as highly.

This lack of understanding of consultation is found in other countries
as well. Many Australian EPSs have also shifted their focus from an
assessment-based to a consultation-based model of service delivery. But
they also experience a lack of cohesion in understanding among
stakeholders regarding the meaning and process of consultation
{[}@bell\_perceptions\_2013{]}. However, some EPs do not report this
problem. In the U.S.A., those who work with school psychologists (as EPs
are called) show a greater consistency of understanding of consultation
in schools. There is also a larger evidence base for the efficacy of
their form of consultation. This is because consultation as practised in
the U.S.A. is almost exclusively Conjoint Behavioural Consultation
(CBC). CBC is defined as ``a strength-based, cross-system
problem-solving and decision-making model wherein parents, teachers, and
other caregivers or service providers work as partners and share
responsibility for promoting positive and consistent outcomes related to
a child's academic, behavioural, and social--emotional development''
{[}@sheridan\_conjoint\_2007, p.25{]}. CBC has been shown to be
effective for CYP over a wide range of settings and for many presenting
problems {[}@sheridan\_randomized\_2017{]}. This hegemony of practice
allows for a consistent definition and implementation, and is likely one
of the reasons consultation in the U.S.A. is better understood and
valued by stakeholders {[}@reddy\_clinical\_2000{]}. It also means its
efficacy can be assessed more readily, such as by conducting a
randomised control trial conducted by @sheridan\_randomized\_2012.

However, the lack of understanding by key stakeholders (SENCOs,
teachers, and parents) may not truly reflect the modern day conception
of consultation in the U.K. The research reporting these findings are
roughly 15 years old. It is reasonable to presume stakeholders have
become more familiar with it, given how prevalent it is. A more recent
paper exploring this is @cording\_study\_2011. For this work, the
authors interviewed 10 school personnel (such as Head teachers and
teachers) and 9 EPs in a Welsh Local Authority (LA). The aim was to
elicit their understanding of the kinds of work they believed EPs engage
in. There was a general alignment between the views of the school
personnel and the EPs themselves. But the school personnel stated they
greatly valued the EP's expertise in diagnosing and alleviating
presenting problems. This shows that despite there being a shared
understanding of what EPs do, there is still a divide in what
stakeholders value about EP work.

\hypertarget{what-are-consultants-views-on-consultation}{%
\subsubsection{1.1.4 What are consultant's views on
consultation?}\label{what-are-consultants-views-on-consultation}}

The vast majority of EPs have a positive view of consultation, with the
Local Offer literature from many EPS stating their model of service
delivery is consultation, such as Kensington \& Chelsea
{[}@royal\_borough\_of\_kensington\_\&\_chelsea\_educational\_2019{]}.
Some EPs believe they provide a unique method of working through their
use of consultation {[}@ashton\_what\_2006{]}. @ashton\_what\_2006 sent
questionnaires to both schools and EPs asking for them to comment on the
work EPs engage in. 22 questionnaires (out of 58) were returned from
mainstream primary schools and eight (out of nine) EPs completed the
questionnaire. `Statutory assessment work', `Closed tests' and `Expert
role' were the most frequently provided parts of EP work that were
classed as unique by SENCOs. Few or no other agencies were judged by
SENCOs as providing a similar or the same service as EPs. `Individual
assessment and intervention' and `Consultation' were the aspects of EP
work the EPs themselves believed were unique to them, as no other
service provided these. This shows a clear disparity between the views
of EPs and key stakeholders within consultation (SENCOs). They also
reported that SENCOs typically valued more traditional EP work, such as
individual assessment and giving advice. The SENCOs did not value
consultation, nor give evidence they had a complete understanding of it.
However, these results should be interpreted with caution. The small
sample limited to one LA undermines our ability to generalise the
results to a wider context. It also only collected data from mainstream
primary schools, further limiting the scope of these results. Yet one of
the main results, namely the lack of understanding regarding the nature
of consultation) has been replicated by research in other school
settings {[}@dennis\_so\_2004; @ofarrell\_research\_2018{]}.

\hypertarget{what-are-the-main-features-of-consultation}{%
\subsubsection{1.1.5 What are the main features of
consultation?}\label{what-are-the-main-features-of-consultation}}

Once a common understanding of what the stakeholders believe
consultation to be has been created, an analysis of the common features
of consultation can occur. The following section will evaluate the
relevant literature regarding the features of consultation.
@henderson\_exploration\_2013 used focus groups with clusters of SENCos
across a small LA to gain an understanding of their beliefs about the
mechanics of consultation. The researchers sat in on five different
Primary SENCo Network meetings and worked to elucidate their views. They
presented the participants with statements about parts of the
consultation process. Their task was to sort them depending on how often
they believed the statements to be a part of a consultation. The mostly
commonly given features of consultation were: discussing issues with
relevant parties; information gathering; and it being a reflexive
process with a focus on collaboratively crafting solutions. They also
conducted semi-structured interviews with EPs, children who were
receiving EP involvement, and their parents. It being a collaborative
and problem-solving process, with a focus on solutions, and the
development of positive working relationships between those involved
were the two main themes. The use of focus groups to identify SENCO
beliefs regarding the nature of consultation and the interviews with the
stakeholders allows comparison between the stakeholder's expectations
and the reality of consultation. However, given the researchers did not
directly observe consultations but relied on self-report, the
conclusions that can be drawn regarding how consultations occur are
weakened. This is because of the disparity between self-reports of
behaviour and real-world instances of behaviour
{[}@argyris\_theory\_1992{]}.

@kennedy\_educational\_2008 thematically analysed the voice recordings
of 17 EP-teacher consultations. These individual case studies
{[}@robson\_real\_2015{]} were supplemented by a pre-consultation
questionnaire completed by EPs to establish their espoused theory for
consultation. A comparison could then be made between the recordings and
participants' self-report to see whether their espoused theory aligned
with the recorded behaviours. The authors report a high correspondence
between the EPs espoused theory and theory in practice as EPs
predominantly engaged in behaviours dictated by their espoused theory.
The most common behaviours by EPs were working collaboratively,
typically with those most involved (predominantly parents) using either
Solution-Focused approaches or problem-solving analysis.
Solution-Focused approaches are characterised by greater interest in the
solutions to presenting problems rather than the problem itself. It
views the client as capable of solving their own problems with a changed
mindset, facilitated by the EP, through identifying times when the
severity of the problem is reduced or it is not present, termed
`exceptions' {[}@rhodes\_solution\_2004{]}. Problem-solving analysis is
related to behavioural consultation {[}@bergan\_behavioral\_1990{]} and
is divided into four stages: problem identification, problem analysis,
treatment implementation, and treatment evaluation
{[}@sheridan\_school\_2000{]}. Those involved with the young person,
such as teachers, are involved throughout
{[}@kennedy\_effective\_2009{]}. By recording the consultations, the
authors could gather data from a larger number of consultations than
they could have if they sat in on every consultation. However, there was
a low granularity of analysis. The researchers only assessed whether
features of the espoused theory appeared at all during the consultation.
Thus, the analysis only shows that during a consultation, EPs brought in
ideas from their espoused theory at least once. There was no analysis of
how frequently the ideas appeared and when during the consultations. It
therefore cannot tell us how great a part these concepts from the
espoused theory played in the consultation, merely that they were
present.

@nolan\_process\_2014 observed seven consultations between five EPs, a
teacher, and at least one parent. A week later, the researchers
conducted semi-structured interviews with all EPs and teachers and some
of the parents. Several key themes arose from the observations and
interviews. These were: empowering those involved in the consultation;
working collaboratively; the importance of each participant in the
consultation recognising the valuable knowledge from others; reviewing
outcomes; and EPs using their expertise to support others (without
emphasising their role as the ``expert''). The use of both observation
and interview generates a lot of data about these 7 consultations,
giving a very detailed understanding of the process. It also allows
corroboration between data collection methods. However, the small sample
size limits the generalisability of the findings. These results
therefore need to be replicated with different configurations of
participants and in other school settings.

@ofarrell\_research\_2018 found teachers appreciated consultation as
they felt empowered to support the pupils who had been referred.
According to @jones\_refocusing\_1990, this empowering of consultees
rather than fixing the consultees problems or simply giving advice, is
part of the definition of consultation. @dennis\_so\_2004 found that EPs
and SENCOs saw several key issues relating to the successful
implementation of a consultation model: the EP having a detailed
knowledge of the system (school) they were working in; addressing issues
at multiple levels (rather than just on the individual level); positive
relationships between the EP and SENCO; and empowering staff to
successfully fix their problems, rather than doing it for them.

@dickinson\_consultation\_2000 \& @munro\_angles\_2000 examined how
consultation had been implemented in their EPS (Lincolnshire and
Buckinghamshire respectively). Behaviours and approaches which helped
support the successful implementation of consultation across both EPS's
included: having purposeful conversations; EPs using their psychological
knowledge during consultations; and all parties involved deciding on
interventions as well as reviewing past and current interventions.
Factors that were only reported in @munro\_angles\_2000 included:
engaging in preventative work; improving outcomes; and engaging in
multi-level collaborative work. Unfortunately, these papers are based on
the author's reflections on the implementation of consultation in their
LA and do not give the views of anyone else or provide much in the way
of data to support their findings. Readers must therefore take them at
their word.

This research builds on a previous piece of work by the lead researcher.
This first work explored what EPs believe the key features of a
consultation are and what happened in an initial consultation between at
least an EP and a school staff member. This was done through a novel
questionnaire asking EPs to rank features of consultation according to
their importance and thematically analysing transcripts of
consultations. During the consultations, the two most frequent features
of consultation were `Understanding the presenting problem' and `Working
together to come up with solutions'. EPs rated these as core features of
consultation in the questionnaire, as well as improving outcomes for
young people. Whilst this research assessed what EPs believe the core
features are and what the features are of an initial consultation, the
small sample size (3 observed consultations and 8 EPs completing the
questionnaire) means the results are hard to generalise beyond the
immediate consultations.

Although these studies typically only focused on a small number of
participants, the consistency in results allows fundamental features of
consultation to be gleaned. The studies also cover a wide range of EPS,
so the results are not limited to a specific region. This increases the
generalisability of the findings. However, despite these consistencies,
there is still a great deal of heterogeneity in consultation models and
practice. EPs can state they are engaging in consultation, but without
more information or a previously established working relationship, those
involved (parents, teachers, etc.) are unlikely to know what to expect
with a consultation. An arguably more serious consequence is that
assessing the efficacy of consultation is very difficult. If
consultations are not ergodic due to the very wide range of features,
any assessment of consultation may not be valid for consultations
performed by an individual EP. Therefore, assessing the efficacy of
consultations is difficult. This is against the backdrop of EPs working
within `traded services' {[}@lee\_exploration\_2017{]}, where the
ability to demonstrate efficacy is highly valued. It therefore behoves
EPs to gain an understanding of the consistent features of consultation.
This will allow some assessment of which features are correlated with
improved outcomes for CYP.

\hypertarget{assessing-the-efficacy-of-consultation}{%
\subsubsection{1.1.6 Assessing the efficacy of
consultation}\label{assessing-the-efficacy-of-consultation}}

There have been calls for assessing the efficacy of EP work for decades,
such as @cline\_quality\_1994, but this has become even more important
since the almost complete shift to `traded services'
{[}@national\_college\_for\_teaching\_and\_leadership\_educational\_2014{]}.
`Traded services' marks a shift in funding towards existing service
organisations needing to generate income from schools (seen as
customers) to either partially or fully financially support itself
{[}@woods\_preparation\_2014{]}. Many EPs feel a pressure from schools
to both provide something tangible for customers and to demonstrate the
effectiveness of their work, so schools buy their services again
{[}@lee\_exploration\_2017{]}. EPs are also expected to use
evidence-based tools and to critically evaluate their practice
{[}@british\_psychological\_society\_quality\_2015, Standard 4.8;
@health\_\_care\_professions\_council\_standards\_2015, Standard 12.1{]}
as part of the requirements of practising as an EP. It is therefore very
important for EPs to understand what aspects of consultation are
effective in eliciting change.

Measuring such change is difficult. As @kennedy\_educational\_2008
notes, due to the complex nature of the interactions between consultant,
consultee, and client it is difficult to decide what to measure and how
to do so. Several methods have been put forward but none have gained
ascendancy yet. One method used by some EPS
{[}@hampshire\_eps\_how\_2010{]} is the Target Monitoring Evaluation
{[}@dunsmuir\_evidencebased\_2009{]}. Target Monitoring Evaluation (TME)
is based on Goal Attainment Scaling (GAS), which was developed by
@kiresuk\_goal\_1968 to evaluate the outcomes of mental health
interventions. TME is a streamlined form of GAS, with the added
advantage of increased granularity in evaluating progress in relation to
expected progress. TME involves the negotiated development of SMART
goals (specific, measurable, achievable, realistic, and time limited)
between the EP and the consultees. TME forms were completed by both EPs
at two Local Authorities and assistant EPs in one County. During the
initial consultation, after the goals had been decided upon, each
participant rated how far along on a 10-point scale the child currently
was towards each goal. They then stated how far they expected the child
to be when they had their review consultation. 6-8 weeks later, during
the review consultation, each participant rated how far the child had
actually progressed, which was compared with how far they were predicted
to progress. Interviews were conducted with teachers, SENCOs, and
headteachers, who gave positive feedback on the easy and efficiency of
the process, as well as how the tool helped focus on setting of targets.
Two of those interviewed had experience with GAS and stated they
preferred TME. Focus groups of EPs and assistant EPs elicited positive
views towards the tool, as well as considerations of implementation.

This pilot study gives evidence for TMEs efficacy in assessing progress
in response to EP intervention. However, the limited detail provided in
the report means we do not have a fine grained understanding of the
strengths and weaknesses of the tool. @monsen\_evaluation\_2009 assessed
the efficacy of assistant EPs using TME and focus groups with
stakeholders. Both measures found assistant EPs to be beneficial to EP
work. This work was conducted in one EPS, therefore the generalisability
of the findings are limited. TME is a quantitative assessment of
efficacy and the focus groups produced qualitative data, comparison of
efficacy between the two measures is difficult. We therefore cannot draw
strong conclusions about the validity of TME when assessing educational
psychology work.

There have been a few studies which have attempted to compare TME with
other quantitative measures of change, such as @connor\_target\_2010. In
this thesis, the author compared TME with other, more established forms
of progress measurement in domains like reading, such as the York
Assessment of Reading Comprehension (YARC). They report that there was
broad agreement between the TME and other forms of assessment; when
other forms of assessment found improvement, this was reflected in the
reported change through the TME forms. However, while TME may be useful
for identifying progress in individual children, it was not clear how it
could be used to assess the quality of the work from the EP. There were
also some difficulties regarding the use of it, as there was
disagreement between some consultants and consultees regarding target
setting and the voice of the child.

A recent paper {[}@eddleston\_using\_2018{]} comparing different
consultation evaluation tools excluded TME because it did not reach the
inclusion criteria. Here, TME's streamlined nature counted against it as
it was not sufficiently thorough enough to be evaluated. This means
there is limited evidence for its efficacy as a tool. However, as
@dunsmuir\_evidencebased\_2009 states, ``the strengths of GAS are
maintained but the TME system is more streamlined and user friendly'' (p
67). We can therefore have increased confidence in the validity of TME
as a measure of change, given that GAS has been shown to be a useful
tool {[}@roach\_goal\_2005{]} and it shares fundamental similarities.

\hypertarget{local-offer-literature}{%
\subsubsection{1.1.7 Local Offer
literature}\label{local-offer-literature}}

To gain an understanding of what EPs at different LAs understood
consultation to be, the Local Offer literature was examined. This
information was found on the LA's websites and detailed what services
the EPS provided. Despite almost all services having moved to a
consultation-based service delivery
{[}@dinkmeyer\_consultation\_2016{]}, over a third of LAs did not
explicitly mention consultation. Of those that did, the most commonly
cited feature was working with relevant parties, such as teachers. The
second most common was improving outcomes for the CYP, with the
importance of looking for solutions (including the use of
Solution-Focused approaches) also being mentioned frequently. What this
shows is that for the LAs that mention it, the EPs working there have
explicitly stated the importance of collaborating with those closest to
the CYP and the necessity of improving the CYP's outcomes.

\hypertarget{context-and-rationale}{%
\subsection{1.2 Context and rationale}\label{context-and-rationale}}

This project was greatly shaped by the coronavirus (COVID-19) pandemic
and the subsequent response by the British Government. Because of this,
the research was conducted during unprecedented circumstances. All EPs
(and workers in general) had to work remotely from home. This presented
unique concerns for how EPs worked, as they were not allowed to see any
adults or CYP in person. Several documents, such as
@bhardwaj\_rapid\_2020, and one piece of research
{[}@aep\_survey\_2020{]} were disseminated drain this time, detailing
guidance as to how EPs can work ethically within the context of a
lockdown. This included conducting consultations using either phones or
video call software, such as Microsoft Teams or Zoom. There were
concerns regarding the safety and privacy of Zoom technology
{[}@paul\_zoom\_2020{]} so Teams was encouraged by many EPS. This
shifted the manner in which the research could be conducted:
consultations could not be observed in person and many EPs were not
engaging in consultation. The timeline of the research was changed as a
result, with the observation of consultations pushed back to September
2020 when it was hoped they would resume by. There was also a broadening
of the parameters of the research. Questions exploring the impact of the
lockdown were posed to interviewees and a questionnaire was designed and
disseminated to explore a wide range of views towards consultation and
how it had been affected by the pandemic.

The purpose of this research was it gain an insight into what happens
during a joint school-family consultation, as well as which features
correlate with rated changes towards agreed goals. Having a more
fine-grained understanding of when different feature are seen and how
frequently would provide valuable insight into what are the core feature
of a consultation. This could then be cross-referenced with the ratings
of progress as measured by TME. Because of the global pandemic, a
related question was explored regarding the use of technology when
conducting consultations. Gaining an understanding of what the core
features of consultation is allows EPs to understand what is essential
for a consultation to occur. This will inform the professions response
to the lockdown, subsequent lockdowns, and EP work in general. EP
beliefs regarding how consultation can be performed and their
experiences changing their work can give guidance as to how EPs should
use consultation in future, beyond the scope of the COVID-19 pandemic or
any future pandemics.

\hypertarget{research-questions}{%
\subsection{1.3 Research questions}\label{research-questions}}

Given the lack of strong theory in this area of research, research
questions were developed but statistical hypotheses could not be drawn.
Thus, it is exploratory research {[}@kimmelman\_distinguishing\_2014{]}.
The research questions are:

\begin{enumerate}
\def\labelenumi{\arabic{enumi}.}
\item
  What do EPs believe are the core features of a joint school-parent
  consultation?
\item
  Which features of consultation correlate with increased progress
  towards agreed goals?
\end{enumerate}

\hypertarget{methodology}{%
\section{2 Methodology}\label{methodology}}

\hypertarget{epistemology-and-research-paradigm}{%
\subsection{2.1 Epistemology and research
paradigm}\label{epistemology-and-research-paradigm}}

To explore these questions, a mixed methodology was employed, making use
of quantitative and qualitative research methods. It was informed by a
scientific realism epistemology. Scientific realism can help approach
difficult problems in social science as it takes into account the
complexity of the situation in which they occur
{[}@house\_realism\_1991{]}. It can be viewed as a pragmatic approach
{[}@robson\_real\_2015{]} as it is less concerned with philosophical
dualisms, such as rationalism versus empiricism, and more with practical
considerations of issues and potential solutions
{[}@johnson\_mixed\_2004{]}. Mixed methodology aligns with a pragmatic
approach as it is not beholden to one method of exploring a research
topic; it sees the benefits of both for exploring a research question in
different ways {[}@denscombe\_communities\_2008{]}. Multiple methods of
inquiry were employed because it is generally believed using different
means to explore research questions brings greater rigour
{[}@creswell\_research\_2003{]}. Data can be triangulated with one
another, with evidence corroborating, refuting, or adding nuance to each
other and increasing confidence in ones findings
{[}@munafo\_robust\_2018{]}. Mixed methodology research designs can be
divided along a key dimension: paradigm emphasis
{[}@johnson\_mixed\_2004{]}. This refers to whether one strand of the
research (quantitative or qualitative) is given greater emphasis during
analysis. Because equal weight was placed on both forms of inquiry, this
was an `equal weight' paradigm emphasis piece. An explicit account of
the ways in which the qualitative and quantitative arms of the research
relate to one another will be given {[}@denscombe\_communities\_2008{]}.

\hypertarget{participants}{%
\subsection{2.2 Participants}\label{participants}}

Ethical approval was obtained from UCL the Institute of Education's
Ethical Committee. The inclusion criteria for all three arms of the
research was: an EP or TEP who used consultation as part of their
practice. There were no requirements as to how frequently or recently it
had to be used, nor experience or location. Nor were there requirements
around the definition of consultation; just that EPs believed themselves
to be engaging in consultation. This was to try and elicit a wide a
range of views on consultations from practising EPs. For the interview
and observation, participants were recruited via the researcher's EPS.
Convenience sampling was therefore used. This was because participant
recruitment for the observation was judged to be difficult and the
researcher would have greater success by asking participants they
already had a professional relationship with. The interview also
recruited participants by sharing a call for participation on a popular
mailing list for EPs and other education professionals (EPNET) and
social media (Twitter). Participants were also asked to share the call
for participants with other EPs at their work. This was the method used
for recruiting participants to the questionnaire as well. Thus, a
mixture of convenience and snowball sampling {[}@robson\_real\_2015{]}
was employed for these two arms of the research.

\hypertarget{materials}{%
\subsection{2.3 Materials}\label{materials}}

All materials, along with raw data, are released under a CC-BY license,
thus allowing re-use of materials and improving reproducibility and
transparency {[}@nosek\_scientific\_2012{]}. They can be accessed at:
\url{https://osf.io/nra86/} in the `Methods' folder. Almost all
materials used were Free/Libre and Open Source Software
{[}@stallman\_floss\_2016{]}.

\hypertarget{interviews}{%
\subsubsection{2.3.1 Interviews}\label{interviews}}

A semi-structured interview format was used because an interview
schedule was developed (Appendix 1) which served as a checklist of areas
to be explored with a given question order and wording. However, the
order and wording was allowed to change given the flow of the interview.
Additional questions were used to further develop a interviewee's answer
{[}@robson\_real\_2015{]}. The interviews were of the focused type as
the questions centred around the key theme of consultation
{[}@merton\_focused\_1990{]}. Probes (interview devices to elicit more
information) were employed by the researcher to further develop the
interviewee's responses. To achieve this, `laddering questions'
(questions phrased in a variety of ways asking for the interviewee to
expand on their answer) and `summarising techniques' (summarising what
has just been said by the interviewee to prompt more information), as
well as `addition probes' to maintain the flow of the conversation
{[}@zeisel\_inquiry\_2006{]}. All interviews were recorded with an Honor
10 lite phone. The anonymous transcript was thematically analysed using
the software NVivo .

\hypertarget{observation}{%
\subsubsection{2.3.2 Observation}\label{observation}}

The quantitative arm of the research involved systematic observations of
joint home-school consultations with an EP. Thus, it was a naturalistic
observation as the participants were observed in their typical
environment without any interference from the researcher
{[}@vigliocco\_tip\_tongue\_2001{]}. A systematic observation was chosen
as it helps overcome the often recorded discrepancy between what people
say they do and how they behave in the real-world. This has been
reported in such wide-ranging fields as smartphone use
{[}@andrews\_beyond\_2015{]} to driving behaviours
{[}@kaye\_comparison\_2018{]}. They involve the development of a coding
scheme (Appendix 2) to identify categories over the course of a set
period of time. The categories are defined and operationalised prior to
data collection {[}@croll\_systematic\_1986{]}. They were derived from
the relevant literature and were mutually exclusive. The categories were
limited to what was explicitly said. Models of consultation, such as
Solution-focused and problem-analysis, were broken down into their
constituent observable parts, such as exploring strengths and
identifying exceptions. Event sampling was used as the absolute and
relative frequency of events was of interest {[}@robson\_real\_2015{]}.
A sequence record was also used to provide information as to the order
in which the features were seen, thus providing information about
transitions {[}@robson\_real\_2015{]}. Time sampling was not chosen so
no events were missed because they fell outside of the time intervals.
However, the length of time each feature occurred for was lost. Whilst
this information would be valuable to see how long each feature lasted
for, rather than just how frequently it occurred, it was decided that
the risk of missing feature due to the researcher focusing on correctly
marking the time of each feature outweighed the benefit of gaining that
information.

To explore the relationship between the features and the change in the
goals for each consultation, Qualitative Comparison Analysis (QCA) was
used. It is characterised as a ``small-N-many variables'' approach.
Configurational Comparative Methods: Qualitative Comparative Analysis
(QCA) and Related Techniques: ``QCA techniques allow the systematic
comparison of cases, with the help of formal tools and with a specific
conception of cases.''

``In the process of configurational comparative analysis, the researcher
engages in a dialogue between cases and relevant theories. Indeed, the
choice of the variables (conditions and outcome) for the analysis must
be theoretically informed. In this sense, there is a deductive aspect to
QCA; however, QCA techniques can also be used more inductively, gaining
insights from case knowledge in order to identify the key
``ingredients'' to be considered (Rihoux, 2003, 2006; Rihoux \& Lobe,
2009)."

``QCA techniques allow for ``conjunctural causation'' across observed
cases. This means that different constellations of factors may lead to
the same result (equifinality)"

``By using QCA, the researcher is urged not to specify a single causal
model that best fits the data, as one usually does with statistical
techniques, but instead to determine the number and character of the
different causal models that exist among comparable cases (Ragin,
1987).''

for the statistical programming language R {[}@r\_core\_team\_r\_2017{]}

\url{https://www.ncbi.nlm.nih.gov/pmc/articles/PMC1089061/pdf/hsresearch00022-0148.pdf}:
``to assess the sufficiency of a combination of causal conditions, the
researcher selects cases with a given combination of conditions and then
evaluates whether or not these cases display the same, or roughly the
same, outcome.''

\hypertarget{questionnaire}{%
\subsubsection{2.3.3 Questionnaire}\label{questionnaire}}

A questionnaire was designed using Qualtrics to explore consultation as
conducted both during the lockdown and prior to it. The questions were
informed by the answers to the interview questions, as recommended by
{[}@gehlbach\_measure\_2011{]}. Due to the fact the way answer options
are presented can bias results {[}@schwarz\_self\_reports\_1999{]}, the
construction of the question and answers was guided by the best practice
recommendations from @gehlbach\_measure\_2011. Quantitative questions
explored what the key features of consultation are (both before and
during the lockdown), what kinds of work EPs engaged in (before and
during the lockdown), how they have found the changes to their work
(with Likert scale ratings), and how much different types of work have
been affected by the lockdown. Questions suggesting different features
of consultation were based on the scientific literature, for example
@dennis\_so\_2004; @dickinson\_consultation\_2000;
@farrell\_developing\_2006; @henderson\_exploration\_2013;
@kennedy\_educational\_2008; @munro\_angles\_2000; \&
@nolan\_process\_2014, and the material on the Local Offer websites.
These were the same features used in the observation schedule.
Open-ended questions will be used to explore the changes to EP
consultations as a result of the lockdown and their views towards
technologically mediated consultations. This is because there is no
published literature to suggest what EPs may experience, given the
unprecedented nature of the present circumstances. It is therefore best
to give participants the opportunity to respond how they wish, without a
narrowing of options by the questionnaire.

\hypertarget{procedure}{%
\subsection{2.4 Procedure}\label{procedure}}

Prior to data collection, it was decided the quantitative arm would be
conducted first, starting in March 2020 and continuing until March 2021.
Interviews would be conducted in the autumn of 2020. Thus, a concurrent
triangulation design would be employed {[}@creswell\_research\_2003{]}.
Both the quantitative and qualitative arms of the research would be
conducted simultaneously and independently. The results were to be
compared to see whether the conclusions drawn align with one another.
This was done for practical rather than philosophically informed
reasons. It was agreed beforehand that collecting observation data would
be more difficult, as finding consultations with all the required
participants who were also willing to be observed is unlikely.
Consultations with both the teacher and parent present are less likely
to occur than consultations with just one of them, given the specific
multiplication rule of probability {[}@grinstead\_introduction\_1997{]}.
It was therefore felt that having a longer window of opportunity to
collect data was the reasonable course of action.

However, due to the COVID-19 pandemic all in person consultations were
cancelled across the U.K. to comply with the government-mandated
lockdown {[}@cabinet\_office\_staying\_2020{]}. Whilst many EPs offered
consultations to their respective schools, most found they delivered far
fewer consultations during the lockdown than usual. Those who delivered
consultations typically did so via the phone, eliminating any chance of
observation by the researcher. In response, data collection for the
interviews was brought forward to start in March 2020 and observations
of consultations would occur once consultations could be observed by the
researcher. The research was therefore adapted to use a sequential
transformative design {[}@creswell\_research\_2003{]}. This type of
mixed methodology involves one method preceding the other. Either the
qualitative or the quantitative arm of the research project is conducted
first. The methodology does not require one be used before the other, so
practical reasons may determine the order of research. The results from
both strands are interpreted together, with one informing the other.

\hypertarget{interviews-1}{%
\subsubsection{2.4.1 Interviews}\label{interviews-1}}

Interviews were originally planned to be in person with EPs in the
researcher's EPS. However, because of the global pandemic, all
non-essential in person meetings were banned. They were therefore
switched to video or phone call interviews. Because of the sudden
increase in proficiency and willingness of many EPs to use phone and
video call technology, the parameters of the participant recruitment for
the interviews was widened to all EPs. This decision was made because of
a desire to increase the number of participants and thus the range of
views on consultation.

Semi-structured focused interviews were used to elicit EP views with
regards to the core features of consultation, the barriers to effective
consultation, how their consultation work has changed in response to the
lockdown, and the advantages and disadvantages of this new way of
working. 27 EPs were interviewed using a mixture of phone and video call
technology. Data collection took place between 31/03/2020 and
28/05/2020. All interviews were recorded and an anonymous transcript
made. These transcripts were thematically analysed, which involves the
identification of themes through ``careful reading and re-reading of the
data'' {[}@rice\_qualitative\_1999{]}. A mixed or hybrid thematic
analysis approach {[}@fereday\_demonstrating\_2006{]} was employed. This
incorporates inductive and deductive thematic analysis. Inductive
thematic analysis is driven primarily by the data
{[}@boyatzis\_transforming\_1998{]} and deductive thematic analysis is
theory-driven with codes derived from said theory
{[}@crabtree\_template\_1992{]}. The a priori codes identified were
developed from the scientific and Local Offer literature. Semantic
themes (that which is explicitly said) were found and analysed
{[}@boyatzis\_transforming\_1998{]}.

\hypertarget{observation-1}{%
\subsubsection{2.4.2 Observation}\label{observation-1}}

After gaining informed consent from all participants, the researcher
observed the consultation unfold as normal. The researcher will use the
observation schedule to mark when and how frequently different features
occur. These will then be summed. Immediately after the conclusion of
the consultation, each participant (EP, school staff member, and
parent/guardian) was asked to collectively identify 2-3 goals for the
CYP to work towards. This was done using a TME form. Participants rated,
on a scale of 1-10, where the CYP currently was towards that goal (by
writing the letter `B' for `baseline' next to the number) and where they
expected them to be in 6-8 weeks (by writing the letter `E' by the
number). In 6-8 weeks time, participants would be contacted by the
researcher via email to rate how far along the CYP had progressed
towards that goal. This judgement was represented by the letter `A' (for
`actual') along the same rating scale. This data will be summarised with
the median ratings for each category (`baseline', `expected', and
`actual') presented. The summed features would be tallied against the
progress made for each observed consultation. This will be calculated by
subtracting the `baseline' rank from the `actual' as research suggests
most TME forms report a positive change as a result of the consultation
{[}@dunsmuir\_evidencebased\_2009; @monsen\_evaluation\_2009{]}. Whilst
the independent variable of frequency counts of features is continuous,
the dependent variable of the reported progress is ordinal. Therefore, a
Spearman rank-order correlation will be used
{[}@field\_discovering\_2012{]} to measure the correlation between
features of consultation and reported change in outcomes for the CYP.

\hypertarget{questionnaire-1}{%
\subsubsection{2.4.3 Questionnaire}\label{questionnaire-1}}

A questionnaire was designed to supplement the findings from the
interviews. By identifying key ideas from the interview answers, these
could be explored with a larger sample by using a questionnaire. Data
collection took place between 26/05/2020 and 04/06/2020. Descriptive
statistics of the types of features employed during consultations will
be reported, along with how much of an impact the lockdown has had on
different kinds of work and the changes made to work during the
pandemic. The open text questions exploring participant views towards
the use of technology when conducting consultations and how their work
has been impacted by the lockdown were thematically analysed. Inductive
thematic analysis was used to explore the semantic themes
{[}@boyatzis\_transforming\_1998{]}. Inductive analysis is data driven
and thus codes are derived from the data, rather than pre-determined
codes being used to analyse the data {[}@guest\_applied\_2012{]}. The
question assessing how much each type of work has been affected will be
plotted to check the distribution of responses. Likert rating scales
produce ordinal data and the responses are unlikely to be normally
distributed, therefore models which assume normally distributed
continuous data are inappropriate {[}@liddell\_analyzing\_2017{]}. It
will therefore be analysed using a cumulative probit ordinal regression
model {[}@burkner\_ordinal\_2019{]} to see if there differences between
types of work as well as groups, such as role. A probit model will be
used as the latent variable which the question is seeking to measure
(how greatly various kinds of work have been affected by the lockdown)
is assumed to be normally distributed
{[}@mccullagh\_regression\_1980{]}.

To see whether the lockdown has affected the prevalence of different
features of consultation, a two-way within subjects ANOVA will be used.
This is suitable because the data is proportional
{[}@mangiafico\_summary\_2016{]}. The features will be clustered into
subgroups: Solution-Focused, Problem analysis, Organisation and
knowledge; and Valuing everyone (see Appendix 3 for a breakdown of the
features into the subgroups). The subgroups will be analysed
independently with the other factor (time) common to all subgroup
analyses. Time will have two levels (before and during lockdown).
Responses which did not include any data for questions relating to the
features of consultation or how their work had been impacted by the
pandemic were excluded.

\hypertarget{reflections-on-pilot}{%
\subsection{2.5 Reflections on pilot}\label{reflections-on-pilot}}

\hypertarget{interview-schedule}{%
\subsubsection{2.5.1 Interview schedule}\label{interview-schedule}}

The interview was piloted with a Trainee Educational Psychologist (TEP)
to check for flow and whether the interviewees understood the questions.
The TEP commented on the definition of ``features'' in question 5. The
word was changed to ``features'' and a clarification statement will be
provided, along with a definition if necessary.

\hypertarget{observation-schedule}{%
\subsubsection{2.5.2 Observation schedule}\label{observation-schedule}}

To establish inter-rater reliability (IRR), an anonymous transcript of a
previously recorded consultation was analysed for feature using the
observation schedule. Three raters, including the researcher, assessed
the transcript for feature of consultations in their relative order.
Intraclass correlations {[}@shrout\_intraclass\_1979{]} were calculated
between the three raters. The relative frequency of each category was
calculated for each rater and compared with one coder's (the
researchers) results. Because frequency counts were used, intraclass
correlations (ICC) were suitable as the data is continuous. To calculate
ICC, four factors must be decided upon prior to calculation
{[}@hallgren\_computing\_2012{]}. A two way model was used because the
raters weren't randomly selected from the population. Given that a
non-timed sequence record design was chosen for the observation
schedule, good IRR was defined as consistency in the ratings because it
was more important that raters provide scores that are similar in rank
order. A single measures ICC was calculated because the reliability of
the other two raters needed to generalise to ratings of one coder (the
researcher). And finally, a mixed model was used because the raters were
not randomly chosen from a population. This model was applied using the
irr package {[}@gamer\_irr\_2019{]} in R. This produced an ICC of 0.471
which, according to guidelines provided by @Cicchetti1994 are
`substantial'.

The categories are defined and operationalised prior to data collection
{[}@croll\_systematic\_1986{]}. They were derived from the relevant
literature and were mutually exclusive. This was to increase the
reliability as it reduces the chances of observations being coded
differently according to the interpretation of an observer. To further
reduce risks to reliability, the categories were limited to what was
explicitly said. This was done to minimise the amount of inference the
researcher had to use when deciding whether a category was observed
{[}@croll\_systematic\_1986{]}. Models of consultation, such as
Solution-focused, were broken down into their constituent observable
parts, such as exploring strengths and identifying exceptions, so the
categories were more fine-grained and which specific features of the
models were used during consultations.

\hypertarget{questionnaire-2}{%
\subsubsection{2.5.3 Questionnaire}\label{questionnaire-2}}

14 TEPs piloted the questionnaire. They identified a few questions which
could be misinterpreted (``What will you do differently when things go
back to normal'') and questions which would benefit from an explanation
as to how the answering mechanic worked. They also identified additional
types of work EPs could engage in during the lockdown. Overall the
feedback was positive, with particular focus on the breadth of questions
and the inclusion of changes to practice during the current lockdown.
One TEP reported that questions 17 and 18 could be interpreted in
different ways: do EPs think they should change their consultations
compared with how they were conducted prior to the lockdown or how they
are done now? Because the focus of the question is about what they will
do differently as a result of their experiences, the question was
changed to reflect this. A comment was raised about the order of the
options when selecting the magnitude of the impact on different kinds of
work due to the lockdown. It was decided it would be in ascending order
as this makes more intuitive sense when reading from left to right.

\hypertarget{results}{%
\section{3 Results}\label{results}}

NEED INTRODUCTION TO SECTION

MAYBE THEMATIC MAP?

\documentclass{article}
\usepackage{tikzit}
\input{sample.tikzstyles}

\begin{document}

A tikz picture as an equation:
\begin{equation}
  \tikzfig{fig1}
\end{equation}

A centered tikz picture:
\ctikzfig{fig1}

\end{document}

\hypertarget{interviews-2}{%
\subsection{3.1 Interviews}\label{interviews-2}}

30 EPs of varying roles and locations were interviewed. Participant's
roles included TEPs, maingrade EPs, specialist EPs, senior EPs, and
Principal EPs. The participants worked in locations such as London,
Yorkshire, Wales, and the Republic of Ireland. Thematic analysis
identified 32 inductive codes, as well as the 15 deductive codes,
relating to what features EPs believed were effective for consultation.
6 codes were identified for what made said features effective (see
Appendix XXX for all codes and definitions). These were combined to
create 8 themes: Buy-in, Conditions, Context, Strengths-based, Shared
understanding, Intervention, Future facing, and EP skills and knowledge.
These could then be combined to create two super themes: Internal
factors and External factors (see Appendix XXX for a complete mapping of
codes to themes and super themes).

\hypertarget{buy-in}{%
\subsubsection{3.1.1 Buy-in}\label{buy-in}}

This theme related to the importance of EPs creating a bond with those
involved, including the consultee(s) and other school staff members not
directly involved in the consultation, and using this relationship to
facilitate change.

\hypertarget{collaborative}{%
\paragraph{3.1.1.1 Collaborative}\label{collaborative}}

One of the fundamental and most oft cited features for creating buy-in
was making consultation collaborative. Within the consultation, this was
achieved through a variety of factors. One of the key ones was making
sure there was equal participation, such that everyone had a voice and
different perspectives were heard: ``effective consultation shouldn't
being a meeting where one person dominates, whether that may be a
psychologist or anyone else'' (Interview 11) and ``it's like we're all
involved, we're all at the same level, we just come at it from a
different perspective'' (Interview 7).

As a result of there being equal participation, there is a greater
chance that everyone involved has the same understanding of the
situation and the CYP: ``to bring everyone together, and to co-create
and co-construct a shared narrative'' (Interview 11). Misunderstandings
can be cleared up (Interview 5) and these help everyone feel involved in
the process and ensure that the consultation is collaborative. The
creation of a shared narrative can also include the the creation of a
shared agenda. This helps guide the consultation so it is more effective
as it is meeting the needs of those involved and everyone agrees to it:
``I think a really fundamentally important part of that consultation is
ensuring that we do have that shared agenda; we know why we're there
together and we all agree what we're doing there together'' (Interview
24) and ``to arrive at a joint action plan, joint for the school and the
parents, school are always involved as well, so it's more
collaborative'' (Interview 10).

This shared agenda can be established by identifying what everyone is
hoping to get from the consultation:

it would always start with a question about what are your best hopes
from our meeting together? What are your best hopes from our work
together? Because if we don't start with that question, erm, then we
don't know where we're trying to get to. (Interview 27)

By working collaboratively with those involved, EPs can facilitate
collaboration between the home and school. This can potentially support
both by helping maintain morale and creating a sense of shared
responsibility:

there is something that goes on often, not always, in the room when
you've got the family, and school together, the, you do you do bring
that sense of, `We are working on this together; you are not alone
school in this, you are not alone parents in this, we are doing this
together'. (Interview 5)

\hypertarget{contributions-valued}{%
\paragraph{3.1.1.2 Contributions valued}\label{contributions-valued}}

A related code, and one which can facilitate a collaborative
consultation, is the idea that everyone who is present in the
consultation should feel able to contribute. Not only this, but they
need to believe that what they say will be taken on board:

where I would like to think that their views, their knowledge, their
understanding is just as valid as mine\ldots{} we are equal participants
in this (Interview 13).

equal participation, you know, as far as possible, or that everybody
participates and that everybody feels valued, everybody feels that what
they had to say is useful (Interview 20).

This can help give power to those who may not typically have it in the
school environment, thus helping create a more level playing field and
therefore a more collaborative consultation: ``schools are by nature
very hierarchical. So if you've got a TA they're often not seen as the
same as, you know, a SENCO or a head teacher's views but in that
situation they are'' (Interview 1).

\hypertarget{encouraging-engagement}{%
\paragraph{3.1.1.3 Encouraging
engagement}\label{encouraging-engagement}}

Removing power dynamics within a consultation was seen by many
participants as an important part of the EPs role within consultation.
This formed part of the code `EP encouraging engagement'. The EP must
try and create a space so no consultee feels intimidated and in which
all relevant people can contribute, even if they cannot physically be
present:

the psychologist trying to level power dynamics is a really key, a
really key part of any consultation and that erm that's in relation to
ourselves, as a professional with a doctorate normally, but also in
relation to the family and the teacher, or the family and the school.
(Interview 2)

balance of people's voices in the rooms. So, erm, making time for those
that might not be able to be present in the meeting to hear their views
and voices. (Interview 27)

This code related to any effort by the EP to attempt to include the
voices of the relevant parties. One of the ways that this is through
``active listening'' (Interview 1). A key idea related to the EP
facilitating others to participate:

I'm there to help facilitate the group in thinking about ways forward.
(Interview 15)

giving a space where people can listen to other people's perspectives,
then you take away the bulk of what it is that you're, erm, using to try
and make a difference. (Interview 21)

Not only does the EP need to facilitate others, but also challenge
potentially harmful narratives and navigate difficult situations:

being careful and being prepared to challenge. (Interview 25)

sometimes a kind of mediation role because it's, we work in complex and
messy situations. And it's not always that people are going to agree, or
even really want to hear what they have to say. So there's that kind of
control in the, the floor that happens in a consultation, which doesn't
happen in other types of conversation. (Interview 3)

\hypertarget{rapport}{%
\paragraph{3.1.1.4 Rapport}\label{rapport}}

This ability to challenge is related to another core feature, which is
the development of a rapport with those involved in consultations.
Within the consultation, an EP must quickly develop a rapport so that
the consultees feel comfortable talking about potentially difficult
topics:

trust and credibility and shared mutual respect, I think are at the core
of any consultation. You know, they value what I offer because I'm in
touch and the fact they get on well with me, that almost therapeutic
relationship. (Interview 7)

built up that trust and sense of safety, that it's okay to express their
worries, that you can get quite a lot of information. (Interview 10)

The EP needs to not only develop a rapport with those involved, but
encourage relationships between consultees: ``building attuned
interactions in a meeting with parents, with teachers, and then
hopefully between them as well. It just kind of gets everyone on the
same page, hopefully gets everyone pointing in the right direction''
(Interview 30). This is especially important when relationships between
the home and school have broken down:

Because if you don't have that, you know, sometimes you have a breakdown
between parents and the school, the relationship, you know, in a way
that, you can be a person in between, and try and get that working
through that, erm, which is, you know, a key feature of consultation,
that, you know, you're working in some difficult situations, erm, and if
there's a breakdown, in the relationship between both, erm, it's a way
of trying to bring it back together. (Interview 4)

Several EPs talked about the importance of having a good relationship
with the school. A good relationship between the school (generally
understood to mean at least the SECNCos and potentially Senior
Leadership Team) helps consultation to be more effective:

if it's going to be successful model in a school, I think the need is
that, actually, you know, time for the EP to build a relationship with
the school is important. (Interview 23)

The reason the relationship is crucial for improving consultation is
that when the EP has developed a good relationship with the school and
they are mutually supporting one another, it is easier to create an
environment which fosters collaboration:

when you know the school especially, and they're supporting you in
supporting the parents and the staff to do that, then you see it a lot
more'' (Interview 1).

Schools are often hesitant to adopt consultation as the main method of
EP work: some of the SEN schools that I work with have a very rigid way
of seeing the EP role and what we do, and they're, they're view is, more
often than not, my role as an EP is to go in, do an assessment, write a
report, and that's it. Er, so in those instances, I find it much harder
to sell consultation as a, as a model. (Interview 11)

However, several EPs spoke of using their relationship with the school
to change how they approach EP work and what the EP can do in the
school:

once you build a relationship with schools, and you've been working in
it, you can shift things, you can move things around, to, you know,
working with a bit more control, getting them to see how, you know, it
can be more effective, working with consultation, not doing just lots of
assessments. (Interview 4).

That's how you change it. I think that the relationship is super
important. (Interview 23)

\hypertarget{ep-view-of-consultation-and-consultee-view-of-consultation}{%
\paragraph{3.1.1.5 EP view of consultation and Consultee view of
consultation}\label{ep-view-of-consultation-and-consultee-view-of-consultation}}

An important feature of consultation that relates to rapport is the
understanding that the consultees, EP, and school as a whole have
towards consultation. How the EP and consultees view consultation can
have a large impact on a consultation and its efficacy. A belief shared
by many interviewees was that ``both parties, kind of, know how
consultation works'' (Interview 24) and this ``might depend on people's
constructs of what consultation is'' (Interview 29). Interviewees had an
overwhelmingly positive view of consultation, highlighting its
versatility and alignment with their values:

consultation, I think, is a, is a framework with the complexity that
matches the complexity of the concerns that are being raised. Erm, we're
looking at concerns at an individual and a group and a systemic level
(Interview 21).

I don't think you can be inclusive without using a consultative model
(Interview 25).

Though many interviewees identified the value of consultation and the
importance of clearly understanding it and what it involves, many also
pointed out that there is a large heterogeneity of practice among EPs:
``I think that concept of what a consultation is will vary from one EP
to another'' (Interview 24). There are also EPs who do not value it and
prefer a more traditional style of assessing children and then writing a
report. As one interviewee said: ``I know there's a lot of EPs out there
that continue to work in that way and I think, I think that's one of the
barriers to shifting more to a consultation framework'' (Interview 17).
One interviewee, who had recently attended a course on consultation
provided by their EPS, stated:

I'm not sure a lot of EPs really understand what it is. Being able to
communicate that\ldots{} even on that consultation course that I
mentioned I went on, I was really surprised that people, people very
open and very honest, and they said, `We've been saying we've been using
consultation, but we actually have not. We've realised now that we
haven't really been using consultation'. (Interview 22)

This makes it difficult for consultees to gain a clear understanding of
what consultation is and has led a few EPs to call for clearer
communication and ``being better at communicating\ldots{} what it is and
what it can do'' (Interview 22). One of the reasons it is important
consultees understand what consultation means is so they can see the
value in it. Many interviewees described how some of the schools they
work in do not appreciate it fully:

if I could click my fingers and change something on a systemic level, it
would be the attitude toward consultation because I I really view them
as an investment. If you invest in a consultation, you're going to get
better work and and outcomes. Whereas, sometimes they can be viewed as
an expensive hurdle you have to get over to get a standardised score.
(Interview 2)

I think there are some schools that, erm, have a negative view of
consultation. Because of that. It's, it's more complex procedure I
think, people realise. (Interview 10)

I think we need to educate our schools more about `This is what the
process is', because we say in sales blurb `We do a consultation' and,
erm, and then the schools are still stuck in that, kind of, old way of
thinking. (Interview 28)

A recurring comment centred around the differences between primary and
secondary schools, with primaries typically being more willing to engage
with them:

most primary SENCOs are very open to whatever I suggest. And they're
quite open to different ways of working, as long as they have a report
to use as evidence, er, for EP involvement, so it has that element of of
a tick box. But most primary schools are very open to different ways of
looking, I would say, but secondaries definitely aren't. (Interview 18)

\hypertarget{ep-view-of-eps-and-consultee-view-of-eps}{%
\paragraph{3.1.1.6 EP view of EPs and Consultee view of
EPs}\label{ep-view-of-eps-and-consultee-view-of-eps}}

Another relevant strand to the different perceptions of consultations is
how the consultees view EPs and their role. Several interviewees talked
about how they were viewed as gatekeepers to resources or as someone who
would fix the situation independently of any work by the consultees:

the associations that staff or parents can have of us as being, kind of,
the deciders of resources. So we will go in and we will say, and we will
think we are there to support to think about what we can do for this
child, and they will think we are coming in to say `Yes you can have any
EHCP' or `Yes you can have extra money'. (Interview 1)

if school are new to that way of working and they are used to having an
EP come in and, sort of, tell them what to do. I do notice that
sometimes there's a bit of confusion, er, especially from some teachers
who are, `Why are you asking me, aren't you supposed to tell me what I
need to do'? (Interview 11)

How receptive a school is to consultation as a way of working ``very
much comes down to the school's view of my role'' (Interview 14).

How the consultees view the EP can be changed in the consultation
itself: ``You're modelling how psychologists think\ldots{} they might
think a psychologist is on a pedestal or whatever, but you're modelling
that psychologists are like everybody else'' (Interview 7).

Being able to read body language was identified by a few EPs as being
important for facilitating engagement:

You try to do an online meeting, you lose the gesticulations, you lose
the, er, being able to point at things or being able to, you know, look
at their faces better and realise, `Oh, they're not understanding, I
need to change the way I'm explaining it' or something. I think you lose
so much because it's that non-verbal feedback that you get, that allows
you to know where you are at with the relationship, to know the way you
can develop within that consultation. (Interview 24)

However, this was not universal. A few EPs found that using
technologically-mediated (tech) consultations did not lead to a decrease
in quality of the relationship. One EP experienced her consultees asking
for telephone consultations and that these were effective. (Interview
16)

WHAT MAKES IT EFFECTIVE

These features (collaborative, EP facilitating engagement, rapport,
contributions valued) helps facilitate one of the key mechanisms by
consultation is effective: an increased chance of realistic
recommendations and outcomes. If the ideas generated are more
co-constructed and built on shared knowledge, they are more likely to be
feasible:

it also allows for reality, so if you've, you know, hopefully you're not
getting ideas or strategies that are completely unworkable. So it should
be based within the practice of the class teacher. So it isn't, you
know, somebody coming in and going, `Well, you need to do this three
times a day with, you know, dah, dah, dah, dah, dah'. (Interview 21)

the feedback we get from parents that things are very grounded in
reality, that the ideas that we're talking about makes sense because
they come from a position of understanding and making sense of whatzver
is being brought into the room and, sort of, helping to manage some of
the complexity. (Interview 27)

Collaboration was identified as a key factor not only because it
increased the consultee's willingness to engage but because it increased
the chances of the recommendations being put in place:

I think if you have a really good consultation and you can actually
problem solve together, and the people that you're consulting with,
actually come up with some of the ideas, then it's much more likely for
those interventions to happen. (Interview 20)

By having a collaborative consultation where people´s voices are heard
and their expertise and contributions valued, this helps create a sense
of consultee ownership. They are more likely to buy into the process and
are therefore more likely to feel some responsibility for supporting the
CYP. Through the EP empowering those involved by identifying skills
already present, consultees feel better able to support the CYP and feel
more motivated.

Buy-in was facilitated by the EP not taking an expert stance and
creating a collaborative and sharing environment for the consultees to
explore their thoughts.

\hypertarget{intervention}{%
\paragraph{Intervention}\label{intervention}}

By creating a safe space for exploration and potential challenging,
another key feature of consultation can happen: the consultation acting
as an intervention itself. This can be done

POTENTIALLY USE ELSEWHERE The EP is gathering and summarising the ideas
and saying, `Given what we've discussed, and the ideas we've heard so
far, what is going to make most sense for this young person and what's
going to make most difference'? And then it's getting the ideas from the
people. (Interview 27)

\hypertarget{section}{%
\paragraph{}\label{section}}

\hypertarget{questionnaire-3}{%
\subsection{3.2 Questionnaire}\label{questionnaire-3}}

68 EPs with a wide range of roles, including TEPs and Principal EPs,
from all areas of the U.K. completed the questionnaire. Preliminary
analysis of the data shows that joint school-parent consultations were
one of the most frequently performed types of work prior to the
lockdown, along with parent consultation, individual assessment, and
training. Many stated they could no longer engage in most of their
typical types of work, such as observation, individual assessment, and
joint school-parent consultations. Most stated they found the changes
`very challenging'. A majority of respondents replaced face to face
meetings with telephone calls and some replaced them with video calls.
Some respondents stated they found it harder to elicit the views of CYP
when working remotely.

\hypertarget{qca}{%
\subsubsection{QCA}\label{qca}}

No pair-wise simplifications could be made as there were no
consultations which saw change which differed by only 1 feature.

\includegraphics{Thesis_files/figure-latex/practice graph-1.pdf}
\includegraphics{Thesis_files/figure-latex/practice graph-2.pdf}

\includegraphics{Thesis_files/figure-latex/Thematic map-1.pdf}
\includegraphics{Thesis_files/figure-latex/Thematic map-2.pdf}

\hypertarget{appendices}{%
\section{Appendices}\label{appendices}}

\hypertarget{appendix-1}{%
\subsection{Appendix 1}\label{appendix-1}}

\begin{enumerate}
\def\labelenumi{\arabic{enumi})}
\tightlist
\item
  What is your role?
\item
  How do you define consultation? What does it mean to you?
\item
  What key words would you use?
\item
  How often have you engaged with consultation?
\item
  What history of consultation training do you have?
\item
  Does your current EPS value consultation/operate a consultation-based
  service?
\item
  Why do you use consultation?
\item
  What do you believe are the key features of a consultation? What needs
  to be present for it to be more than a conversation?
\item
  What features do you most frequently see (what is seen may be
  different what they believe is effective)?
\item
  What do you believe are the key features of an effective consultation
  (including examples)?
\item
  What makes them effective?
\item
  How could consultations be more effective?
\item
  What are the barriers to effective consultation?
\item
  If you could not use consultation, what work would you use instead?
\item
  What is the unique contribution of consultation?
\item
  What has changed with regards to your consultation work during
  lockdown?
\item
  How have you found this change?
\item
  Advantages/disadvantages?
\item
  Will you do anything differently after this is over?
\item
  Should the service/EPs as a whole do things differently?
\end{enumerate}

\hypertarget{appendix-2}{%
\subsection{Appendix 2}\label{appendix-2}}

\begin{longtable}[]{@{}ll@{}}
\toprule
\begin{minipage}[b]{0.47\columnwidth}\raggedright
Categories\strut
\end{minipage} & \begin{minipage}[b]{0.47\columnwidth}\raggedright
Definition\strut
\end{minipage}\tabularnewline
\midrule
\endhead
\begin{minipage}[t]{0.47\columnwidth}\raggedright
Info gather\strut
\end{minipage} & \begin{minipage}[t]{0.47\columnwidth}\raggedright
Fact finding or discussion of non-key concern(s).\strut
\end{minipage}\tabularnewline
\begin{minipage}[t]{0.47\columnwidth}\raggedright
Suggesting solutions\strut
\end{minipage} & \begin{minipage}[t]{0.47\columnwidth}\raggedright
The EP volunteering a solution to the presenting concern.\strut
\end{minipage}\tabularnewline
\begin{minipage}[t]{0.47\columnwidth}\raggedright
CYP strengths\strut
\end{minipage} & \begin{minipage}[t]{0.47\columnwidth}\raggedright
Any discussion of the CYP's positive qualities: attributes, personality,
actions, etc.\strut
\end{minipage}\tabularnewline
\begin{minipage}[t]{0.47\columnwidth}\raggedright
Discussing what's already working\strut
\end{minipage} & \begin{minipage}[t]{0.47\columnwidth}\raggedright
Discussion (including evaluation) of any intervention/change which has
improved the current situation for the CYP.\strut
\end{minipage}\tabularnewline
\begin{minipage}[t]{0.47\columnwidth}\raggedright
Everyone's contributions valued\strut
\end{minipage} & \begin{minipage}[t]{0.47\columnwidth}\raggedright
Consultees giving their view on something e.g.~presenting hypotheses,
suggesting solutions, or the EP explicitly acknowledging someone for
their contribution. Not just the consultee(s) speaking/giving an answer
to a factual question.\strut
\end{minipage}\tabularnewline
\begin{minipage}[t]{0.47\columnwidth}\raggedright
Understanding presenting problem\strut
\end{minipage} & \begin{minipage}[t]{0.47\columnwidth}\raggedright
Discussion of any aspect of the main presenting concern(s) including
scope, environmental factors, exceptions, etc. and why a problem may be
present {[}@sheridan\_school\_2000{]}\strut
\end{minipage}\tabularnewline
\begin{minipage}[t]{0.47\columnwidth}\raggedright
Summarising\strut
\end{minipage} & \begin{minipage}[t]{0.47\columnwidth}\raggedright
The EP saying back what has previously been stated by consultees in the
consultation (potentially building on it but not necessarily).\strut
\end{minipage}\tabularnewline
\begin{minipage}[t]{0.47\columnwidth}\raggedright
Planning implementing treatments\strut
\end{minipage} & \begin{minipage}[t]{0.47\columnwidth}\raggedright
Discussion and agreement between the consultant and consultee on any
interventions that will be implemented to support the CYP
{[}@sheridan\_school\_2000{]}.\strut
\end{minipage}\tabularnewline
\begin{minipage}[t]{0.47\columnwidth}\raggedright
EP using expert knowledge\strut
\end{minipage} & \begin{minipage}[t]{0.47\columnwidth}\raggedright
EP discussing topics which they have knowledge of (from both
professional experience and academic reading) within school psychology
theory and practice.\strut
\end{minipage}\tabularnewline
\begin{minipage}[t]{0.47\columnwidth}\raggedright
EP explaining role\strut
\end{minipage} & \begin{minipage}[t]{0.47\columnwidth}\raggedright
EP explicitly talking about the work of an EP and its purpose.\strut
\end{minipage}\tabularnewline
\begin{minipage}[t]{0.47\columnwidth}\raggedright
Setting out plan for consultation\strut
\end{minipage} & \begin{minipage}[t]{0.47\columnwidth}\raggedright
Discussion of what will happen over the course of the
consultation.\strut
\end{minipage}\tabularnewline
\begin{minipage}[t]{0.47\columnwidth}\raggedright
Ideas for future EP work\strut
\end{minipage} & \begin{minipage}[t]{0.47\columnwidth}\raggedright
Discussion of potential work an EP can do in the future, such as
consultation, assessment, observation, etc.\strut
\end{minipage}\tabularnewline
\begin{minipage}[t]{0.47\columnwidth}\raggedright
Empowering individuals\strut
\end{minipage} & \begin{minipage}[t]{0.47\columnwidth}\raggedright
Any comments or questions which aim to increase the skills of the
consultees (teachers, parents, SENCOs, etc.)/upskilling consultees so
they can solve their problems {[}@nolan\_process\_2014{]}.\strut
\end{minipage}\tabularnewline
\begin{minipage}[t]{0.47\columnwidth}\raggedright
School knowledge\strut
\end{minipage} & \begin{minipage}[t]{0.47\columnwidth}\raggedright
Any comments or questions which increase understanding of how the school
works.\strut
\end{minipage}\tabularnewline
\bottomrule
\end{longtable}

\hypertarget{appendix-3}{%
\subsection{Appendix 3}\label{appendix-3}}

\begin{longtable}[]{@{}ll@{}}
\toprule
\begin{minipage}[b]{0.34\columnwidth}\raggedright
Level\strut
\end{minipage} & \begin{minipage}[b]{0.60\columnwidth}\raggedright
feature\strut
\end{minipage}\tabularnewline
\midrule
\endhead
\begin{minipage}[t]{0.34\columnwidth}\raggedright
Solution-focused\strut
\end{minipage} & \begin{minipage}[t]{0.60\columnwidth}\raggedright
Suggesting solutions; Highlighting the strengths of the CYP; Discussing
what is already working; Exploring exceptions; Suggesting ideas for
future EP work.\strut
\end{minipage}\tabularnewline
\begin{minipage}[t]{0.34\columnwidth}\raggedright
Problem analysis\strut
\end{minipage} & \begin{minipage}[t]{0.60\columnwidth}\raggedright
Fully understanding the presenting problem; How to implement the
interventions\strut
\end{minipage}\tabularnewline
\begin{minipage}[t]{0.34\columnwidth}\raggedright
Organisation and knowledge\strut
\end{minipage} & \begin{minipage}[t]{0.60\columnwidth}\raggedright
Gathering information; Summarising; Using knowledge; Setting out a plan;
Explaining what EPs do; School knowledge\strut
\end{minipage}\tabularnewline
\begin{minipage}[t]{0.34\columnwidth}\raggedright
Valuing everyone\strut
\end{minipage} & \begin{minipage}[t]{0.60\columnwidth}\raggedright
Everyone contributing; empowering those involved\strut
\end{minipage}\tabularnewline
\bottomrule
\end{longtable}

\hypertarget{references}{%
\section{References}\label{references}}

\end{document}
